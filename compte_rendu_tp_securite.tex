\documentclass[12pt,a4paper]{article}

% Packages
\usepackage[utf8]{inputenc}
\usepackage[french]{babel}
\usepackage[T1]{fontenc}
\usepackage{geometry}
\usepackage{graphicx}
\usepackage{fancyhdr}
\usepackage{titlesec}
\usepackage{xcolor}
\usepackage{listings}
\usepackage{float}
\usepackage{hyperref}
\usepackage{tcolorbox}
\usepackage{enumitem}
\usepackage{multicol}

% Geometry settings
\geometry{left=2.5cm, right=2.5cm, top=3cm, bottom=3cm}

% Colors
\definecolor{maincolor}{RGB}{0,51,102}
\definecolor{secondcolor}{RGB}{204,0,0}
\definecolor{lightgray}{RGB}{240,240,240}

% Header and footer
\pagestyle{fancy}
\fancyhf{}
\fancyhead[L]{\includegraphics[height=1cm]{Application ensab + guides/ensab_v2/vendor/img/uh1.png}}
\fancyhead[R]{\includegraphics[height=1cm]{Application ensab + guides/ensab_v2/vendor/img/ensa.png}}
\fancyfoot[C]{\thepage}
\renewcommand{\headrulewidth}{0.5pt}

% Title formatting
\titleformat{\section}{\Large\bfseries\color{maincolor}}{\thesection}{1em}{}
\titleformat{\subsection}{\large\bfseries\color{secondcolor}}{\thesubsection}{1em}{}

% Listings configuration
\lstset{
    basicstyle=\ttfamily\small,
    backgroundcolor=\color{lightgray},
    breaklines=true,
    frame=single,
    numbers=left,
    numberstyle=\tiny,
    showstringspaces=false
}

% Custom boxes
\newtcolorbox{exercicebox}[1]{
    colback=lightgray,
    colframe=maincolor,
    title=#1,
    fonttitle=\bfseries,
    boxrule=1pt
}

\newtcolorbox{answerbox}{
    colback=white,
    colframe=secondcolor,
    title=Réponse,
    fonttitle=\bfseries,
    boxrule=1pt,
    height=8cm
}

\newtcolorbox{screenshotbox}{
    colback=white,
    colframe=maincolor,
    title=Screenshot,
    fonttitle=\bfseries,
    boxrule=1pt,
    height=10cm
}

\begin{document}

% Page de garde
\begin{titlepage}
    \centering

    % Logos
    \begin{minipage}{0.45\textwidth}
        \centering
        \includegraphics[width=0.8\textwidth]{Application ensab + guides/ensab_v2/vendor/img/uh1.png}
    \end{minipage}
    \hfill
    \begin{minipage}{0.45\textwidth}
        \centering
        \includegraphics[width=0.8\textwidth]{Application ensab + guides/ensab_v2/vendor/img/ensa.png}
    \end{minipage}

    \vspace{2cm}

    {\LARGE\bfseries Université Hassan 1er\par}
    \vspace{0.5cm}
    {\Large École Nationale des Sciences\par}
    {\Large Appliquées -- Berrechid\par}

    \vspace{3cm}

    {\Huge\bfseries\color{maincolor} Compte Rendu de TP\par}
    \vspace{1cm}
    {\LARGE Sécurité informatique et cybersécurité\par}

    \vspace{2cm}

    {\large\bfseries Filière :} {\large Génie Informatique\par}
    \vspace{0.3cm}
    {\large\bfseries Semestre :} {\large S7\par}

    \vfill

    \begin{minipage}{0.45\textwidth}
        \flushleft
        {\bfseries Réalisé par :}\\
        ENNAQUI Zakaria
    \end{minipage}
    \hfill
    \begin{minipage}{0.45\textwidth}
        \flushright
        {\bfseries Encadré par :}\\
        Pr Abdelhalim HNINI
    \end{minipage}

    \vspace{1cm}

    {\large Année universitaire : 2025/2026\par}

\end{titlepage}

% Table des matières
\tableofcontents
\newpage

% TP 1: Injection SQL Classique
\section{TP 1: Injection SQL Classique}

\subsection{Objectifs}
\begin{itemize}
    \item Comprendre les vulnérabilités d'injection SQL
    \item Exploiter les failles SQL pour accéder à des données non autorisées
    \item Apprendre les techniques de protection contre les injections SQL
\end{itemize}

\subsection{Environnement de travail}
\begin{exercicebox}{Configuration de l'environnement}
\textbf{Étape 1:} Ouvrir l'application ensab\_v2\\
\textbf{Étape 2:} Naviguer vers le module \texttt{sql\_injection}\\
\textbf{Étape 3:} Accéder à la page de connexion
\end{exercicebox}

\begin{answerbox}
% Décrivez ici les étapes de configuration effectuées
\vspace{7cm}
\end{answerbox}

\subsection{Exercice 1: Identification de la vulnérabilité}

\begin{exercicebox}{Analyse du code source}
\textbf{Tâche:} Examinez le fichier \texttt{login.php} dans le dossier \texttt{sql\_injection}\\
\textbf{Question:} Quelle ligne de code présente une vulnérabilité d'injection SQL?
\end{exercicebox}

\begin{answerbox}
% Écrivez votre réponse ici
\vspace{7cm}
\end{answerbox}

\begin{screenshotbox}
% Insérez votre screenshot du code vulnérable
\centering
\textit{[Insérer le screenshot du code source ici]}
\end{screenshotbox}

\subsection{Exercice 2: Exploitation de l'injection SQL}

\begin{exercicebox}{Bypass d'authentification}
\textbf{Étape 1:} Ouvrir la page de connexion\\
\textbf{Étape 2:} Dans le champ "Username", coller: \texttt{admin' OR '1'='1}\\
\textbf{Étape 3:} Dans le champ "Password", coller: \texttt{anything}\\
\textbf{Étape 4:} Cliquer sur le bouton "Se connecter"\\
\textbf{Question:} Que se passe-t-il? Expliquez pourquoi cette injection fonctionne.
\end{exercicebox}

\begin{answerbox}
% Écrivez votre réponse et explication ici
\vspace{7cm}
\end{answerbox}

\begin{screenshotbox}
% Insérez votre screenshot de l'injection réussie
\centering
\textit{[Insérer le screenshot de l'exploitation ici]}
\end{screenshotbox}

\subsection{Exercice 3: Extraction de données}

\begin{exercicebox}{Utilisation d'UNION SELECT}
\textbf{Tâche:} Utilisez une injection SQL de type UNION pour extraire les informations des utilisateurs\\
\textbf{Payload à tester:} \texttt{admin' UNION SELECT username, password FROM users--}\\
\textbf{Question:} Quelles informations avez-vous pu extraire?
\end{exercicebox}

\begin{answerbox}
% Écrivez les informations extraites
\vspace{7cm}
\end{answerbox}

\begin{screenshotbox}
% Insérez votre screenshot des données extraites
\centering
\textit{[Insérer le screenshot des résultats ici]}
\end{screenshotbox}

\subsection{Solutions et recommandations}

\begin{exercicebox}{Correction de la vulnérabilité}
\textbf{Question:} Proposez des méthodes pour sécuriser le code contre les injections SQL
\end{exercicebox}

\begin{answerbox}
% Écrivez vos recommandations de sécurité
\vspace{7cm}
\end{answerbox}

\newpage

% TP 2: Blind SQL Injection
\section{TP 2: Blind SQL Injection}

\subsection{Objectifs}
\begin{itemize}
    \item Comprendre le concept de Blind SQL Injection
    \item Exploiter les vulnérabilités sans retour direct de données
    \item Utiliser les techniques booléennes et temporelles
\end{itemize}

\subsection{Environnement de travail}
\begin{exercicebox}{Configuration de l'environnement}
\textbf{Étape 1:} Naviguer vers le module \texttt{blindsqli}\\
\textbf{Étape 2:} Accéder à la page de connexion\\
\textbf{Étape 3:} Observer le comportement de l'application
\end{exercicebox}

\begin{answerbox}
% Décrivez vos observations initiales
\vspace{7cm}
\end{answerbox}

\subsection{Exercice 1: Détection de la vulnérabilité Blind SQL}

\begin{exercicebox}{Test de vulnérabilité}
\textbf{Étape 1:} Tester avec un username valide\\
\textbf{Étape 2:} Tester avec: \texttt{admin' AND '1'='1}\\
\textbf{Étape 3:} Tester avec: \texttt{admin' AND '1'='2}\\
\textbf{Question:} Quelle différence observez-vous entre les deux requêtes?
\end{exercicebox}

\begin{answerbox}
% Écrivez vos observations
\vspace{7cm}
\end{answerbox}

\begin{screenshotbox}
% Insérez votre screenshot du test
\centering
\textit{[Insérer le screenshot du test ici]}
\end{screenshotbox}

\subsection{Exercice 2: Exploitation booléenne}

\begin{exercicebox}{Extraction de données caractère par caractère}
\textbf{Tâche:} Utilisez une injection booléenne pour déterminer la longueur du mot de passe\\
\textbf{Payload:} \texttt{admin' AND LENGTH(password)>5--}\\
\textbf{Question:} Quelle est la longueur du mot de passe de l'administrateur?
\end{exercicebox}

\begin{answerbox}
% Écrivez votre processus et résultat
\vspace{7cm}
\end{answerbox}

\begin{screenshotbox}
% Insérez votre screenshot du résultat
\centering
\textit{[Insérer le screenshot ici]}
\end{screenshotbox}

\subsection{Exercice 3: Time-Based Blind SQL Injection}

\begin{exercicebox}{Utilisation de SLEEP()}
\textbf{Payload à tester:} \texttt{admin' AND IF(1=1, SLEEP(5), 0)--}\\
\textbf{Question:} Comment cette technique vous permet-elle d'extraire des informations?
\end{exercicebox}

\begin{answerbox}
% Expliquez le principe et vos résultats
\vspace{7cm}
\end{answerbox}

\begin{screenshotbox}
% Insérez votre screenshot démontrant le délai
\centering
\textit{[Insérer le screenshot ici]}
\end{screenshotbox}

\subsection{Solutions et recommandations}

\begin{answerbox}
% Proposez des solutions de sécurité
\vspace{7cm}
\end{answerbox}

\newpage

% TP 3: Attaque par Force Brute avec Burp Suite
\section{TP 3: Attaque par Force Brute avec Burp Suite}

\subsection{Objectifs}
\begin{itemize}
    \item Comprendre les attaques par force brute
    \item Utiliser Burp Suite pour automatiser les tentatives de connexion
    \item Identifier les vulnérabilités liées aux mots de passe faibles
\end{itemize}

\subsection{Configuration de Burp Suite}

\begin{exercicebox}{Installation et configuration}
\textbf{Étape 1:} Ouvrir Burp Suite\\
\textbf{Étape 2:} Configurer le proxy dans votre navigateur (127.0.0.1:8080)\\
\textbf{Étape 3:} Activer l'interception dans Burp Suite (Proxy > Intercept > Intercept is on)\\
\textbf{Étape 4:} Naviguer vers la page de connexion du module \texttt{bruteForce}
\end{exercicebox}

\begin{answerbox}
% Décrivez le processus de configuration
\vspace{7cm}
\end{answerbox}

\begin{screenshotbox}
% Insérez votre screenshot de la configuration de Burp Suite
\centering
\textit{[Insérer le screenshot de Burp Suite configuré ici]}
\end{screenshotbox}

\subsection{Exercice 1: Capture de la requête HTTP}

\begin{exercicebox}{Interception de la requête}
\textbf{Étape 1:} Activer l'interception dans Burp Suite\\
\textbf{Étape 2:} Tenter une connexion avec username: \texttt{admin} et password: \texttt{test}\\
\textbf{Étape 3:} Observer la requête interceptée\\
\textbf{Question:} Quels sont les paramètres envoyés dans la requête?
\end{exercicebox}

\begin{answerbox}
% Décrivez la structure de la requête HTTP
\vspace{7cm}
\end{answerbox}

\begin{screenshotbox}
% Insérez votre screenshot de la requête interceptée
\centering
\textit{[Insérer le screenshot de la requête ici]}
\end{screenshotbox}

\subsection{Exercice 2: Configuration de l'attaque Intruder}

\begin{exercicebox}{Configuration d'Intruder}
\textbf{Étape 1:} Envoyer la requête vers Intruder (Click droit > Send to Intruder)\\
\textbf{Étape 2:} Dans l'onglet "Positions", sélectionner le paramètre password\\
\textbf{Étape 3:} Dans l'onglet "Payloads", charger le fichier \texttt{poc-common-passwords.txt}\\
\textbf{Étape 4:} Lancer l'attaque (Start attack)
\end{exercicebox}

\begin{answerbox}
% Décrivez le processus de configuration
\vspace{7cm}
\end{answerbox}

\begin{screenshotbox}
% Insérez votre screenshot de la configuration d'Intruder
\centering
\textit{[Insérer le screenshot d'Intruder ici]}
\end{screenshotbox}

\subsection{Exercice 3: Analyse des résultats}

\begin{exercicebox}{Identification du mot de passe}
\textbf{Question:} Quel est le mot de passe trouvé?\\
\textbf{Question:} Comment avez-vous identifié la tentative réussie? (taille de réponse, code HTTP, etc.)
\end{exercicebox}

\begin{answerbox}
% Écrivez vos résultats et analyse
\vspace{7cm}
\end{answerbox}

\begin{screenshotbox}
% Insérez votre screenshot des résultats de l'attaque
\centering
\textit{[Insérer le screenshot des résultats ici]}
\end{screenshotbox}

\subsection{Solutions et recommandations}

\begin{exercicebox}{Mesures de protection}
\textbf{Question:} Quelles mesures de sécurité peuvent empêcher les attaques par force brute?
\end{exercicebox}

\begin{answerbox}
% Proposez des solutions de sécurité
\vspace{7cm}
\end{answerbox}

\newpage

% TP 4: Force Brute avec P4ssHunt3r
\section{TP 4: Attaque par Force Brute avec P4ssHunt3r}

\subsection{Objectifs}
\begin{itemize}
    \item Utiliser l'outil P4ssHunt3r pour les attaques par force brute
    \item Comprendre les techniques d'ingénierie sociale
    \item Générer des dictionnaires de mots de passe personnalisés
\end{itemize}

\subsection{Installation de P4ssHunt3r}

\begin{exercicebox}{Installation}
\textbf{Étape 1:} Ouvrir un terminal ou PowerShell\\
\textbf{Étape 2:} Installer avec: \texttt{pip install p4sshunt3r}\\
\textbf{Étape 3:} Vérifier l'installation: \texttt{p4sshunt3r --help}
\end{exercicebox}

\begin{answerbox}
% Décrivez le processus d'installation
\vspace{7cm}
\end{answerbox}

\begin{screenshotbox}
% Insérez votre screenshot de l'installation
\centering
\textit{[Insérer le screenshot de l'installation ici]}
\end{screenshotbox}

\subsection{Exercice 1: Génération d'un dictionnaire personnalisé}

\begin{exercicebox}{Création d'un dictionnaire basé sur l'ingénierie sociale}
\textbf{Scénario:} L'utilisateur cible se nomme "Mohamed", né en 1990, aime le football\\
\textbf{Commande:} \texttt{p4sshunt3r -n mohamed -b 1990 -k football -o custom\_dict.txt}\\
\textbf{Question:} Combien de mots de passe ont été générés?
\end{exercicebox}

\begin{answerbox}
% Écrivez le nombre et quelques exemples
\vspace{7cm}
\end{answerbox}

\begin{screenshotbox}
% Insérez votre screenshot du dictionnaire généré
\centering
\textit{[Insérer le screenshot ici]}
\end{screenshotbox}

\subsection{Exercice 2: Attaque avec le dictionnaire personnalisé}

\begin{exercicebox}{Utilisation du dictionnaire}
\textbf{Étape 1:} Utiliser le dictionnaire généré avec Burp Suite ou un autre outil\\
\textbf{Étape 2:} Lancer l'attaque contre le module bruteForce (niveau medium)\\
\textbf{Question:} Le dictionnaire personnalisé est-il plus efficace qu'un dictionnaire générique?
\end{exercicebox}

\begin{answerbox}
% Comparez les résultats
\vspace{7cm}
\end{answerbox}

\begin{screenshotbox}
% Insérez votre screenshot des résultats
\centering
\textit{[Insérer le screenshot ici]}
\end{screenshotbox}

\subsection{Solutions et recommandations}

\begin{answerbox}
% Proposez des solutions contre l'ingénierie sociale
\vspace{7cm}
\end{answerbox}

\newpage

% TP 5: Force Brute avec mots de passe générés par ingénierie sociale
\section{TP 5: Force Brute et Ingénierie Sociale}

\subsection{Objectifs}
\begin{itemize}
    \item Comprendre les techniques d'ingénierie sociale
    \item Créer des profils utilisateurs pour générer des mots de passe probables
    \item Analyser la faiblesse des mots de passe prévisibles
\end{itemize}

\subsection{Exercice 1: Collecte d'informations}

\begin{exercicebox}{Profiling de la cible}
\textbf{Scénario:} Vous devez créer un profil pour l'utilisateur "admin"\\
\textbf{Informations à collecter:}
\begin{itemize}
    \item Nom et prénom
    \item Date de naissance
    \item Hobbies et centres d'intérêt
    \item Noms de famille
    \item Animaux de compagnie
\end{itemize}
\textbf{Question:} Où pourriez-vous trouver ces informations? (réseaux sociaux, etc.)
\end{exercicebox}

\begin{answerbox}
% Décrivez les sources d'information
\vspace{7cm}
\end{answerbox}

\subsection{Exercice 2: Génération de mots de passe probables}

\begin{exercicebox}{Création de patterns}
\textbf{Exemples de patterns:}
\begin{itemize}
    \item nom + année: \texttt{mohamed1990}
    \item prénom + date\_naissance: \texttt{ahmed15031985}
    \item hobby + chiffres: \texttt{football123}
    \item nom\_animal + année: \texttt{max2020}
\end{itemize}
\textbf{Question:} Créez une liste de 10 mots de passe probables basés sur le profil
\end{exercicebox}

\begin{answerbox}
% Listez vos mots de passe générés
\vspace{7cm}
\end{answerbox}

\subsection{Exercice 3: Test d'efficacité}

\begin{exercicebox}{Taux de réussite}
\textbf{Question:} Comparez le taux de réussite entre:
\begin{itemize}
    \item Dictionnaire générique
    \item Dictionnaire personnalisé basé sur l'ingénierie sociale
\end{itemize}
\end{exercicebox}

\begin{answerbox}
% Analysez et comparez les résultats
\vspace{7cm}
\end{answerbox}

\begin{screenshotbox}
% Insérez vos graphiques ou tableaux comparatifs
\centering
\textit{[Insérer les comparaisons ici]}
\end{screenshotbox}

\subsection{Solutions et recommandations}

\begin{answerbox}
% Proposez des recommandations pour créer des mots de passe robustes
\vspace{7cm}
\end{answerbox}

\newpage

% TP 6: Attaque par Upload de Fichiers
\section{TP 6: Attaque par Upload de Fichiers}

\subsection{Objectifs}
\begin{itemize}
    \item Comprendre les vulnérabilités liées aux uploads de fichiers
    \item Exploiter les failles de validation pour uploader des fichiers malveillants
    \item Apprendre les techniques de protection
\end{itemize}

\subsection{Environnement de travail}

\begin{exercicebox}{Configuration}
\textbf{Étape 1:} Naviguer vers le module \texttt{upload\_file}\\
\textbf{Étape 2:} Observer la fonctionnalité d'upload\\
\textbf{Étape 3:} Identifier les restrictions éventuelles
\end{exercicebox}

\begin{answerbox}
% Décrivez l'interface d'upload
\vspace{7cm}
\end{answerbox}

\begin{screenshotbox}
% Insérez votre screenshot de la page d'upload
\centering
\textit{[Insérer le screenshot ici]}
\end{screenshotbox}

\subsection{Exercice 1: Upload d'un fichier légitime}

\begin{exercicebox}{Test initial}
\textbf{Étape 1:} Uploader une image légitime (JPG ou PNG)\\
\textbf{Étape 2:} Observer où le fichier est stocké\\
\textbf{Question:} Quelle est l'URL du fichier uploadé?
\end{exercicebox}

\begin{answerbox}
% Écrivez l'URL et vos observations
\vspace{7cm}
\end{answerbox}

\begin{screenshotbox}
% Insérez votre screenshot de l'upload réussi
\centering
\textit{[Insérer le screenshot ici]}
\end{screenshotbox}

\subsection{Exercice 2: Tentative d'upload d'un fichier PHP}

\begin{exercicebox}{Bypass de validation}
\textbf{Étape 1:} Créer un fichier PHP simple:\\
\begin{lstlisting}[language=PHP]
<?php
phpinfo();
?>
\end{lstlisting}
\textbf{Étape 2:} Sauvegarder comme \texttt{shell.php}\\
\textbf{Étape 3:} Tenter de l'uploader\\
\textbf{Question:} L'upload est-il bloqué?
\end{exercicebox}

\begin{answerbox}
% Décrivez le résultat de la tentative
\vspace{7cm}
\end{answerbox}

\begin{screenshotbox}
% Insérez votre screenshot de la tentative
\centering
\textit{[Insérer le screenshot ici]}
\end{screenshotbox}

\subsection{Exercice 3: Techniques de bypass}

\begin{exercicebox}{Bypass d'extension}
\textbf{Techniques à tester:}
\begin{enumerate}
    \item Double extension: \texttt{shell.php.jpg}
    \item Extension null byte: \texttt{shell.php\%00.jpg}
    \item Changement du Content-Type dans Burp Suite
    \item Extension alternative: \texttt{shell.phtml}, \texttt{shell.php5}
\end{enumerate}
\textbf{Question:} Quelle technique a fonctionné?
\end{exercicebox}

\begin{answerbox}
% Décrivez la technique réussie et pourquoi
\vspace{7cm}
\end{answerbox}

\begin{screenshotbox}
% Insérez votre screenshot du bypass réussi
\centering
\textit{[Insérer le screenshot ici]}
\end{screenshotbox}

\subsection{Exercice 4: Exécution du code malveillant}

\begin{exercicebox}{Accès au shell}
\textbf{Étape 1:} Accéder au fichier uploadé via son URL\\
\textbf{Étape 2:} Observer l'exécution du code PHP\\
\textbf{Question:} Quelles informations avez-vous pu obtenir?
\end{exercicebox}

\begin{answerbox}
% Décrivez les informations obtenues
\vspace{7cm}
\end{answerbox}

\begin{screenshotbox}
% Insérez votre screenshot de l'exécution du code
\centering
\textit{[Insérer le screenshot ici]}
\end{screenshotbox}

\subsection{Solutions et recommandations}

\begin{exercicebox}{Mesures de sécurité}
\textbf{Question:} Proposez des mesures pour sécuriser la fonctionnalité d'upload
\end{exercicebox}

\begin{answerbox}
% Listez les mesures de sécurité recommandées
\vspace{7cm}
\end{answerbox}

\newpage

% Conclusion générale
\section{Conclusion Générale}

\subsection{Synthèse des TPs}

\begin{answerbox}
% Rédigez une synthèse des apprentissages de tous les TPs
\vspace{10cm}
\end{answerbox}

\subsection{Compétences acquises}

\begin{exercicebox}{Bilan des compétences}
Listez les compétences techniques acquises durant ces TPs:
\begin{itemize}
    \item[$\square$] Exploitation d'injections SQL classiques
    \item[$\square$] Exploitation de Blind SQL Injection
    \item[$\square$] Utilisation de Burp Suite pour les tests de sécurité
    \item[$\square$] Techniques de force brute
    \item[$\square$] Ingénierie sociale pour la génération de mots de passe
    \item[$\square$] Exploitation des vulnérabilités d'upload de fichiers
    \item[$\square$] Compréhension des mesures de sécurité et protections
\end{itemize}
\end{exercicebox}

\subsection{Recommandations générales de sécurité}

\begin{answerbox}
% Rédigez vos recommandations générales
\vspace{10cm}
\end{answerbox}

\newpage

% Annexes
\section*{Annexes}
\addcontentsline{toc}{section}{Annexes}

\subsection*{Annexe A: Commandes utiles}

\begin{lstlisting}[language=bash]
# Burp Suite - Configuration proxy
http://127.0.0.1:8080

# Installation P4ssHunt3r
pip install p4sshunt3r

# Exemple de payload SQL
admin' OR '1'='1'--
admin' UNION SELECT username,password FROM users--

# Exemple Time-Based Blind SQL
admin' AND IF(1=1, SLEEP(5), 0)--
\end{lstlisting}

\subsection*{Annexe B: Ressources supplémentaires}

\begin{itemize}
    \item OWASP Top 10: \url{https://owasp.org/www-project-top-ten/}
    \item PortSwigger Web Security Academy: \url{https://portswigger.net/web-security}
    \item HackTheBox: \url{https://www.hackthebox.eu/}
    \item TryHackMe: \url{https://tryhackme.com/}
\end{itemize}

\subsection*{Annexe C: Glossaire}

\begin{description}
    \item[SQL Injection] Technique d'injection de code SQL malveillant dans les requêtes
    \item[Blind SQL Injection] Injection SQL sans retour direct de données
    \item[Brute Force] Attaque par essais successifs pour trouver un mot de passe
    \item[Burp Suite] Suite d'outils pour les tests de sécurité des applications web
    \item[Ingénierie sociale] Manipulation psychologique pour obtenir des informations
    \item[Upload vulnerability] Vulnérabilité permettant l'upload de fichiers malveillants
    \item[Web Shell] Interface permettant l'exécution de commandes sur un serveur
\end{description}

\end{document}
